\renewcommand{\theequation}{Eq (\arabic{equation})}
\subsection{Standard Deviation}
Standard deviation is a statistical measure that quantifies the amount of variation or dispersion in a dataset. It measures the average distance between each data point and the mean of the dataset. In other words, it provides a measure of how spread out the data is around the mean.\par
The formula for calculating the standard deviation, denoted by the symbol s for a sample, is as follows:
\begin{equation}
\label{eq:Standard Deviation}
s = \sqrt{\frac{1}{{N-1}} \sum_{i=1}^{N} (x_i - \overline{x})^2} \quad  \dots \dots \dots \dots \dots \dots \dots \dots \dots \dots \dots \dots \dots \dots \dots
\end{equation}

Where:
\begin{itemize}
    \item $x_i$ represents each individual data point in the dataset.
    \item $\overline{x}$ (x-bar) represents the sample mean.
    \item $\Sigma$ denotes the sum of all the values.
    \item $N$ represents the total number of data points in the population or sample.
\end{itemize}

\subsection{Root Mean Square}
Root Mean Square (RMS), also known as quadratic mean, is a statistical measure that represents the square root of the mean of the squared values in a dataset. It is commonly used to quantify the magnitude or average value of a set of numbers, particularly when dealing with varying magnitudes or oscillating values.\par
The formula to calculate the RMS is as follows:
\begin{equation}
\label{eq: Root Mean Square}
RMS = \sqrt{\frac{{x_1^2 + x_2^2 + \ldots + x_n^2}}{{n}}}  \quad  \dots \dots \dots \dots \dots \dots \dots \dots \dots \dots \dots \dots \dots \dots 
\end{equation}


\subsection{Minimum and Maximum}
In time series analysis, the terms "maximum" and "minimum" refer to the highest and lowest values, respectively, observed in a given time series dataset. These values represent the extremes of the dataset and provide important insights into the behavior and range of the variable over time.

\subsection{Slope Sign Change}
Slope sign change refers to the occurrence of a change in the direction of the slope or trend of a dataset. It indicates a transition from a positive slope (increasing values) to a negative slope (decreasing values) or vice versa.

\subsection{Kurtosis}
Kurtosis is a statistical measure that quantifies the shape or peakedness of a probability distribution. It provides information about the tails and outliers of a dataset compared to a normal distribution.\par
The formula for kurtosis on calculating it for a sample is:
\begin{equation}
\label{eq: Kurtosis}
\text{Kurtosis} = \frac{1}{n} \sum_{i=1}^{n} \left(\frac{{X_i - \overline{X}}}{{s}}\right)^4 - 3 \quad  \dots \dots \dots \dots \dots \dots \dots \dots \dots \dots \dots \dots \dots \dots 
\end{equation}
Where:
\begin{itemize}
    \item $n$ represents the sample size.
    \item $\Sigma$ denotes the sum of all the values.
    \item $X$ represents each individual value in the sample.
    \item $\bar{x}$ (x-bar) represents the sample mean.
    \item $s$ represents the sample standard deviation.
\end{itemize}

\subsection{Mean Absolute Deviation}
Mean Absolute Deviation (MAD) is a statistical measure that quantifies the average absolute difference between each data point in a dataset and the dataset's mean. It provides information about the dispersion or variability of the dataset.\par
The formula for Mean Absolute Deviation is as follows:
\begin{equation}
\label{eq: Mean Absolute Deviation}
\text{MAD} = \frac{1}{n} \sum_{i=1}^{n} |X_i - \overline{X}| \quad  \dots \dots \dots \dots \dots \dots \dots \dots \dots \dots \dots \dots \dots \dots \dots \dots
\end{equation}
Where:
\begin{itemize}
    \item $MAD$ represents the Mean Absolute Deviation.
    \item $n$ represents the sample size.
    \item $\Sigma$ denotes the sum of all the values.
    \item $X$ represents each individual value in the sample.
    \item $\bar{x}$ (x-bar) represents the sample mean
\end{itemize}


\subsection{Willison Amplitude}
Willison Amplitude is a statistical measure that quantifies the number of zero-crossings or rapid changes in a time series signal. It is commonly used in signal processing and time series analysis to identify abrupt changes or events in the data.
The formula for Willison Amplitude is as follows:
\begin{equation}
\label{eq: Willison Amplitude}
\text{Willison Amplitude} = \text{Count of} \{i \, | \, |x[i] - x[i-1]| > \text{Threshold}\} \quad  \dots \dots \dots \dots  
\end{equation}
Where:
\begin{itemize}
    
\item i  \text{represents the index of each data point in the time series signal.} 
\item x[i]  \text{represents the value of the signal at index i.} 
\item x[i-1]  \text{represents the value of the signal at index i-1.} 

\end{itemize}

\subsection{Waveform Length}
Waveform Length is a feature used in signal processing and time series analysis to measure the cumulative sum of the absolute differences between consecutive data points in a signal. It provides a measure of the overall variation or complexity of the signal.
\par
The formula for Waveform Length is as follows:
\begin{equation}
\label{eq: Waveform Length}
\text{Waveform Length} = \sum_{i=1}^{n} |i[i] - i[i-1]| \quad  \dots \dots \dots \dots \dots \dots \dots \dots \dots \dots \dots \dots
\end{equation}
Where:
\begin{itemize}
    \item $i[i]$ represents the value of the signal at index $i$.
    \item $i[i-1]$ represents the value of the signal at index $i-1$.
\end{itemize}

\subsection{Mean Absolute Value}
Mean Absolute Value (MAV) is a measure used in signal processing and time series analysis to quantify the average magnitude of a signal. It provides an indication of the signal's overall amplitude or energy.
\par
The formula for Mean Absolute Value is as follows:
\begin{equation}
\label{eq: Mean Absolute Value}
\text{MAV} = \frac{1}{N} \sum_{i=1}^{N} |i[i]| \quad  \dots \dots \dots \dots \dots \dots \dots \dots \dots \dots \dots \dots \dots \dots \dots \dots \dots \dots
\end{equation}
\begin{itemize}
    \item N \textrm{ represents the total number of data points in the signal.}
    \item i[i] \textrm{ represents the value of the signal at index }.
\end{itemize}

\subsection{Average Amplitude Change}
The Average Amplitude Change is a time series feature that quantifies the average rate of change in the amplitude of a signal over a specific time period. It provides information about how the signal's amplitude varies over time.
\par
Here's the formula to calculate the average amplitude change:
\begin{equation}
\label{eq: Average Amplitude Change}
\text{Average Amplitude Change} = \frac{1}{N-W+1} \sum_{t=W}^{N} \left| A[t] - A[t-W] \right| \quad  \dots \dots \dots \dots
\end{equation}
Where:
\begin{itemize}
    \item A[t] \textrm{ represents the amplitude value at time } t.
    \item W \textrm{ is the size of the sliding window or the time period over which you want to calculate the average amplitude change.}
    \item N \textrm{ is the total number of time points in the time series.}
\end{itemize}

\subsection{Zero Crossings}
Zero Crossings is a common feature used in signal processing to quantify the number of times a signal crosses the zero-axis. It provides information about the rate of change and the number of sign changes in the signal.
\par 
The formula for calculating the Zero Crossings of a discrete-time signal is as follows:
\begin{equation}
\label{eq: Zero Crossings}
\text{Zero Crossings} = \sum_{i=1}^{N-1} \mathbb{I}(x[i] \cdot x[i-1] < 0) \quad \dots \dots \dots  \dots \dots \dots  \dots \dots \dots
\end{equation}
Where:
\begin{itemize}
    \item N represents the total number of samples in the signal.
    \item x[i] represents the value of the signal at index i.
    \item I is the indicator function that returns 1 if the condition inside the parentheses is true (indicating a sign change), and 0 otherwise.
\end{itemize}

\subsection{Auto Regression}
Autoregression (AR) is a statistical modeling technique used to analyze time series data by representing a variable as a linear combination of its previous values. It assumes that the current value of a variable can be predicted based on its past values.
\par
The autoregressive model of order p, denoted as AR(p), can be represented by the following formula:
\begin{equation}
\label{eq: Autoregression}
X[t] = c + \sum_{i=1}^{p} \phi_i X[t-i] + \epsilon[t]  \quad \dots \dots \dots  \dots \dots \dots  \dots \dots \dots \dots \dots  \dots 
\end{equation}
Where:
\begin{itemize}
    \item c is a constant term (intercept).
    \item $\phi_i$ denotes the autoregressive coefficients for lags i = 1 to p.
    \item X[t-i] represents the value of the variable at time t minus i.
    \item $\epsilon[t]$ represents the error term or residual at time t, which captures the part of the variable's value that is not explained by the autoregressive terms.
\end{itemize}