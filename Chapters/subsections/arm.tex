\subsection{Base}
The base of a prosthetic arm serves as the foundation or framework upon which the rest of the arm is built. It provides stability, structural support, and a means of attaching and controlling the various components of the prosthetic arm.
\begin{figure}[H]   \centering
	\includegraphics[width=5cm,height=5cm]{"Images/base"}
	\caption{Base of Prosthetic hand}
	\label{fig:Base of Prosthetic hand}
\end{figure}
The above diagram was referenced from \cite{bionicarm} where we specifically used base.stl file.
\par 
\subsection{Forearm}
The forearm of a prosthetic arm is the section that extends from the base (socket or exoskeleton) to the wrist joint or hand attachment. It plays a crucial role in providing support, mobility, and functionality to the prosthetic arm. We decided to fit our servo motor here.
\begin{figure}[H]   \centering
	\includegraphics[width=5cm,height=5cm]{"Images/forearm"}
	\caption{Forearm of Prosthetic hand}
	\label{fig:Forearm of Prosthetic hand}
\end{figure}
The above diagram was referenced from \cite{bionicarm} where we specifically used forearm.stl file.

\subsection{Palm}
The palm of a prosthetic arm refers to the portion of the prosthetic hand that comes into direct contact with objects and performs gripping or grasping functions. It plays a critical role in enabling the user to manipulate objects, perform activities of daily living, and regain functional independence. 
\begin{figure}[H]   \centering
	\includegraphics[width=5cm,height=5cm]{"Images/palm"}
	\caption{Palm of Prosthetic hand}
	\label{fig:Palm of Prosthetic hand}
\end{figure}
The above diagram was referenced from \cite{bionicarm} where we specifically used palm.stl file.

\subsection{Fingers}
Fingers in a prosthetic arm are the components that replicate the functionality of human fingers, enabling the user to perform fine motor tasks and grasp objects. The five fingers have varying length and varying size but all fingers have tip, joint and middle part. You can see the following parts in the figure below:
\begin{figure}[H]   \centering
	\includegraphics[width=5cm,height=5cm]{"Images/fingers"}
	\caption{Fingers of Prosthetic hand}
	\label{fig:Fingers of Prosthetic hand}
\end{figure}
The above diagram was referenced from \cite{bionicarm} where we specifically used finger.stl file.