\chapter{CONCLUSION}
% \section{Conclusion}
In this project, we successfully developed an end-to-end speech-to-text system for the Nepali language using deep learning techniques. By fine-tuning the Wav2Vec model on a Nepali speech dataset, we were able to leverage the power of self-supervised learning to overcome the limitations of low-resource language processing. The system effectively converts spoken Nepali into written text with promising accuracy. 
\par
Our model achieved a Word Error Rate (WER) of approximately 18.5\% on the test set, demonstrating its potential for real-world applications such as voice assistants, automated transcription, and accessibility tools for Nepali speakers. The use of the Devanagari script as output tokens ensured accurate character-level predictions for native language expressions.
The results highlight the importance of using pre-trained models, efficient fine-tuning techniques, and high-quality datasets when building speech recognition systems for underrepresented languages. Despite challenges like data scarcity and dialectal variations, our approach proved effective and scalable.
\par
This project lays the groundwork for future research and development in Nepali ASR systems and opens the door for integration into various practical applications, including education, media, government services, and assistive technology.
% \section{Future enhancement}
% To overcome the limitations produced by our current system, we can enhance it further in future. The future enhancements points are as follows:
% \begin{itemize}
%     \item The whole contro system can be made real time so every movement can be predicted and functioned in real time.
%     \item Currently, we are using 2 channel sensor to collect data, but usage of 8 channel sensor can easily overcome the limitations of 2 channel.
%     \item The material used in 3D printing can be replaced with Silicon rather than thermoplastic material which looks more natural and provides compatibility. 
% \end{itemize}
