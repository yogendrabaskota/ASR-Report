\chapter{INTRODUCTION}

Nepal, a diverse country with over 120 languages spoken across its regions, primarily relies on Nepali as the lingua franca. Nepali, written in the Devanagari script, is used in government, media, and education, serving as a crucial link among the country's different ethnic and linguistic communities. However, despite the importance of Nepali, technological advancement in language processing have been slow, particularly in speech-to-text 
technologies. The project "Nepali Speech to Text using deep learning" aims to bridge this gap by developing a robust, accurate, and user-friendly speech-to-text system tailored for the Nepali language.
\section{Background}
Globally, speech-to-text technology has seen significant advancements, with major tech companies creating highly accurate systems for widely spoken languages such as English, Spanish, and Chinese. These systems have become integral in various applications, from virtual assistants like Siri and Google Assistant to transcription services and accessibility tools for the hearing impaired. However, similar technology for Nepali is still in its infancy\cite{link6}, limited by a lack of comprehensive language databases, research, and development resources. Speech-to-text technology converts spoken language into written text, offering numerous benefits:

- Accessibility: Assists individuals with disabilities, making digital content more accessible. \\ - Efficiency: Enhances productivity by allowing hands-free documentation and faster data 
entry. \\ - Documentation: Provides accurate transcriptions for educational, legal, and media purposes. \\ - Communication: Facilitates communication in professional and personal settings, especially in remote and multilingual environments.
 
% \section{Motivation}

\\\\

% \begin{itemize}
% 	\item The primary motivation behind developing a prosthetic hand is to restore the lost functionality of individuals who have undergone limb amputation.
% 	\item  Prosthetic hands seek to improve the overall well-being of individuals by enabling them to participate fully in society, engage in recreational activities, and maintain a sense of normalcy.
%         \item Prosthetic hands contribute to improving mobility and integration of a person. 
% %	\item To specify the power required for the drive system
	
% \end{itemize}



\section{Problem Statement}
The lack of robust speech recognition systems for the Nepali language poses a barrier to technological accessibility and innovation for Nepali-speaking communities. Existing solutions are often limited in their accuracy, adaptability, and accessibility, failing to cater to the diverse linguistic nuances and socio-cultural contexts inherent in Nepali speech. This presents a pressing need for the development of a tailored speech recognition system capable of accurately transcribing Nepali speech into text, thereby unlocking a myriad of applications and opportunities for Nepali speakers.
\\ \\ 
\\
\section{Objectives}

% The main objective of this project is to study the pattern recognition of EMG signals using many classification algorithms.



\begin{itemize}
	\item Develop a robust Nepali Speech-to-Text Recognition System using Deep Learning techniques and enhance accessibility and usability.  
%	\item Bridge the gap in speech technology for the Nepali language, empowering Nepali speaking communities with innovative and inclusive technology solutions.  
%	\item To specify the power required for the drive system
	
\end{itemize}

\section{Applications}
\begin{enumerate}
    \item \textbf{Voice-Controlled Systems for Nepali Language Users}\\
    Nepali speech-to-text can enable voice assistants, smart devices, and hands-free controls for native speakers. This enhances accessibility for elderly and visually impaired users in Nepal.
    
    \item \textbf{Automatic Transcription for Media and Education}\\
    News broadcasts, interviews, lectures, and online classes in Nepali can be automatically transcribed. This aids in documentation, content indexing, and accessibility for students and professionals.\\

    
    \item \textbf{Government and Public Service Automation}\\
    Public service kiosks, complaint registration, and form filling can be voice-enabled in Nepali. This improves user interaction and inclusivity, especially for people with limited literacy.
\end{enumerate}
% \begin{enumerate}
%     \item To collect and study muscle signal and implement it to classify hand motions and implement in prosthesis for people with no or disfunctional arm.
%     \item To be used clinically for the diagnosis of neurological and neuromuscular problems.

%     \item To be used by patients who have lost their arms due to amputation, congenital conditions, or other injuries.
%     \item To be used in daily activities, work, sports and recreation,  and research and development.

% \end{enumerate}


% \section{Organization of Report}
% The rest of the report is organised as chapter 2 section describing the background theory we used for this project, chapter 3 section describing existing scholarly work and papers related to our project. The section of chapter 4 serves as a
% roadmap that explains how the project was executed, allowing others to understand
% the project. The section of chapter 5 describes the hardware and software components
% that we used in the project.The section of chapter 6 provides a concise summary and
% reflection on the project's overall journey, achievements, and impact.